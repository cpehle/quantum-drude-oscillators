%==============================================================================
%cccccccccccccccccccccccccccccccccccccccccccccccccccccccccccccccccccccccccccccc
%==============================================================================
\documentclass[aps,prb,amsmath,preprint,showpacs,letterpaper]{revtex4}

\usepackage{graphicx}
\usepackage{times}
\usepackage{subfigure}
\usepackage{asymptote}
\usepackage{dcolumn}         %- Align table columns on decimal point
\usepackage{longtable}
\usepackage{multirow}

\newcommand{\grad}{\nabla}
\newcommand{\tdot}{\!\cdot\!}
\newcommand{\bra}[1]{\langle #1|}
\newcommand{\ket}[1]{|#1\rangle}
\newcommand{\braket}[2]{\langle #1|#2\rangle}
\newcommand{\calR}{{\cal R}}
\newcommand{\calY}{{\cal Y}}
\newcommand{\calC}{{\cal C}}
\newcommand{\where}{\text{where}\quad}
\newcommand{\order}[1]{\mathcal{O}\left(#1\right)}
\newcommand{\kA}{{ k_\text{A} }}
\newcommand{\kB}{{ k_\text{B} }}
\newcommand{\lA}{{ l_\text{A} }}
\newcommand{\lB}{{ l_\text{B} }}
\newcommand{\mA}{{ m_\text{A} }}
\newcommand{\mB}{{ m_\text{B} }}
\newcommand{\bi}{{\mathbf{i}}}
\newcommand{\omegaA}{{ \omega_\text{A} }}
\newcommand{\omegaB}{{ \omega_\text{B} }}
\newcommand{\ten}[1]{$\times\!10^#1$}
\newcommand{\myto}{$\!$-}
\newcommand{\br}{{\bf r}}

\newcommand{\erf}{\mathrm{erf}\,}
\newcommand{\erfc}{\mathrm{erfc}\,}

\newcommand{\specialcell}[2][c]{%
  \begin{tabular}[#1]{@{}c@{}}#2\end{tabular}}

%\renewcommand{\thefootnote}{\fnsymbol{footnote}}



%==============================================================================
\begin{document}
\title{Supplementary Material}




\clearpage
%==============================================================================
\section{Detailed Model Potential}

\begin{table}[!ht]
\caption{Detailed Model Parameters}
\begin{tabular}{ccc}
  Symbol & Value & Source / Description  \\ \hline
  $R_{\text{OH}}$ & 0.9572\AA & \multirow{2}{*}{Gas-phase geometry} \\
  $\widehat{\text{HOH}}$ & 104.52$^\circ$   \\ \hline
  q_{\text{H}} & 0.605 e & \multirow{2}{*}{Gas-phase charge moments} \\
  $R_{\text{OM}}$ & 0.2667\AA  \\ \hline
  $\omega$ & 0.6287 $E_h$/$\hbar$& \multirow{3}{*}{\specialcell{Gas-phase polarization and \\dispersion responses}} \\
  m & 0.3656 $m_e$\\
  q_d & -1.1973 e\\ \hline
  $\sigma_d$ & 0.1 a_0 & \multirow{4}{*}{\specialcell{Gaussian charge widths for drudon,\\tether-point (center), H-atom and M-site\\- damp Coulomb force at short range }} \\
  $\sigma_c$ & 1.2 a_0 \\
  $\sigma_\text{H}$ & 0.1 a_0 \\
  $\sigma_\text{M}$ & 0.1 a_0 \\ \hline
  $\kappa_1$ & 2.5 $E_h$ &\multirow{4}{*}{O-O repulsion parameters} \\
  $\lambda_1$ & 1.171802 $a_0^{-1}$ \\
  $\kappa_2$ & 6000 $E_h$ \\
  $\lambda_2$ & 2.820276 $a_0^{-1}$ \\
\end{tabular}
\end{table}

\clearpage
%------------------------------------------------------------------------------
\subsection{Coulomb potential with Gaussian charge width}

For a Gaussian distribution with width $\sigma$, the enclosed volume as a function
of radius is the error function:
\begin{eqnarray*}
  \rho(r) &=& q \tfrac{1}{\sqrt{2\pi}\sigma} \exp\left(-\tfrac{r^2}{2\sigma^2}\right), \\
  \grad^2\phi &=& \rho \implies \\
  \phi(r) &=& \frac{q\:\text{erf}(r/\sqrt{2}\,\sigma)}{r}. \\
\end{eqnarray*}

For two Gaussian distributions interacting, we can use the fact that variances
of independent variables add: 
\begin{eqnarray*}
  \sigma_{12}^2 &\to& \sigma_1^2 + \sigma_2^2, \\
  \phi(r) &=& \frac{q_1 q_2 \: \text{erf}(r/\sqrt{2}\,\sigma_{12})}{r}. 
\end{eqnarray*}

%------------------------------------------------------------------------------
\subsection{Exponential repulsive potential}

This model adds a repulsion potential between the O-atoms, 
with two exponential terms:
\begin{eqnarray*}
  \phi(r) &=& \kappa_1\exp(-\lambda_1 r) + \kappa_2\exp(-\lambda_2 r).
\end{eqnarray*}
Note that $\kappa_i$ and $\lambda_i$ are not related to the dispersion scaling
parameter $\kappa$ nor the polarizability scaling parameter $\lambda$
mentioned in the text.


%\clearpage
%===============================================================================
%\bibliography{suppmat}

%-------------------------------------------------------------------------------
\end{document}
%===============================================================================

